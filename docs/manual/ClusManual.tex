\documentclass[a4paper]{book}
\usepackage{geometry}
\usepackage{ifpdf}
\usepackage{makeidx}
\usepackage{fancyhdr}
\usepackage{fancyvrb}
\usepackage{amsmath, amsthm, amssymb}

\ifpdf
  \usepackage[pdftex]{graphics}
  \pdfinfo{
  /Title    (Clus: A Predictive Clustering System)
  /Author   (J. Struyf)
}
\else
  \usepackage[dvips]{graphics}
\fi

\ifpdf
\usepackage[pdftex,colorlinks=true,pdfstartview=FitV,linkcolor=black,citecolor=black,urlcolor=black]{hyperref}
\else
\usepackage{url}
\newcommand{\phantomsection}[1]{}
\fi

\newtheorem{example}{Example}
\newtheorem{definition}{Definition}[chapter]
\newcommand{\myaddtoc}[1]{\phantomsection{}\addcontentsline{toc}{chapter}{#1}}
\newcommand{\myb}{\discretionary{}{}{}}
\newcommand{\myc}{\discretionary{-}{}{}}
\newcommand{\mmb}{,\discretionary{}{}{}}
\newcommand{\mmc}{,$\discretionary{}{}{\mbox{$\:$}}$}
\newcommand{\R}{\mathbb{R}}

% geometry makes sure page size is correct in .pdf files
\geometry{a4paper,
          centering,
          textwidth = 16cm,
          textheight = 22cm
}

\newcommand{\Clus}{{\sc Clus}}
\newcommand{\TildeB}{{\sc Tilde}}

\setlength{\parindent}{0mm}
\setlength{\parskip}{3mm}

\makeindex

\begin{document}
\thispagestyle{empty}
\begin{centering}
\mbox{}
\vspace{2cm}

{\Huge \Clus{}: A Predictive Clustering System}
\vspace{1cm}

{\Huge User's Manual}
\vspace{3cm}

\mbox{}\hfill\includegraphics{fig/iris-shot}\hfill\mbox{}

\vspace{3cm}
{\Large J. Struyf}

\mbox{}\hfill\today{}\hfill\mbox{}
\end{centering}

\newpage
\pagestyle{fancy}
\tableofcontents

\chapter{Introduction}

\Clus{} is a predictive clustering system. In predictive clustering \cite{Blockeel98:proc}, the data is partitioned into a set of clusters and the data points in each cluster are characterized by a (symbolic) model. 

\begin{figure}
\centering
\includegraphics{fig/3clusters}
\caption{\label{3clusters}A toy example dataset with two numeric attributes $a_1$ and $a_2$, which is partitioned into clusters $C_1$, $C_2$ and $C_3$.}
\end{figure}

\Clus{} currently contains two predictive clustering algorithms: one for learning predictive clustering trees (PCTs) \cite{Blockeel98:proc} and one for learning predictive clustering rules (PCRs) \cite{Zenko06:proc}. In PCTs, the symbolic model is a decision tree, of which each leaf describes a cluster in the partition, and in PCRs, the symbolic model is a rule set and each rule describes one of the clusters.



\chapter{Installing Clus}

\chapter{An Example PCT Run}

\chapter{Clus Settings}

\cleardoublepage
\myaddtoc{References}
\bibliographystyle{plain}
%\bibliography{/home/ml/tex/BIB/.mlbib}
\bibliography{/u/struyf/paper/tex/mlbib/mlbib}

\cleardoublepage
\myaddtoc{Index}
\printindex

\end{document}
